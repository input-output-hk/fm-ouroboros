\documentclass[a4paper,11pt]{article}

\usepackage{typearea}

\usepackage{lmodern}
\usepackage[T1]{fontenc}
\usepackage{textcomp}

\usepackage{amsmath}

\usepackage{isabelle,isabellesym}

\newcommand{\lesssim}{\mathrel{\substack{\textstyle<\\\textstyle\sim}}}
\newcommand{\lessapprox}{\mathrel{\substack{\textstyle<\\\textstyle\approx}}}

\usepackage{pdfsetup}

\urlstyle{rm}
\isabellestyle{it}

\newcommand{\openandsendexample}{
  (\nu y)P
  \overset{\overline{x}\hspace{-0.03em}(\hspace{-.06em}y\hspace{-.12em})}{\longrightarrow}
  Q
}

\begin{document}

\title{A Theory of Labeled Transition Systems\\with Support for Scope Openings}
\author{Wolfgang Jeltsch\\\small\texttt{wolfgang@well-typed.com}}

\maketitle

\tableofcontents

\parindent 0pt\parskip 0.5ex

\section{Introduction}

We present a theory of labeled transition systems, formalized in Isabelle/HOL. Our theory allows
a transition to open scopes by having the label contain binders that bind their respective names
also in the target process. In the $\pi$-calculus, for example, a transition $\openandsendexample$
is possible if $P \overset{\overline{x}y}{\longrightarrow} Q$ is. The latter transition just sends
$y$ along~$x$, but the former first opens the $y$-scope and only then sends $y$ along~$x$. The
channel denoted by~$y$ is accessible in~$Q$ because the $(y)$ in the label $\overline{x}(y)$ binds
$y$ in~$Q$. Certain other process calculi, like $\psi$-calculi and the $\chi$-calculus, even allow
for opening multiple scopes in a single transition.

Our theory of transition systems assumes that binding structures that realize scope openings are
implemented using higher-order abstract syntax (HOAS).

\input{session}

\end{document}
